% ============================================================================
% BSc Computer Science Dissertation
% Reinforcement Learning for Optimizing Retrieval-Augmented Generation
% ============================================================================

\documentclass[12pt,a4paper,oneside]{report}

% ============================================================================
% PACKAGES
% ============================================================================

% Core formatting
\usepackage[utf8]{inputenc}
\usepackage[T1]{fontenc}
\usepackage{lmodern}
\usepackage[margin=2.5cm]{geometry}
\usepackage{setspace}
\onehalfspacing

% Graphics and figures
\usepackage{graphicx}
\usepackage{float}
\usepackage{subcaption}
\graphicspath{{figures/}}

% Tables
\usepackage{booktabs}
\usepackage{tabularx}
\usepackage{longtable}
\usepackage{multirow}

% Math
\usepackage{amsmath}
\usepackage{amssymb}
\usepackage{amsfonts}

% Code listings
\usepackage{listings}
\usepackage{xcolor}

\definecolor{codegreen}{rgb}{0,0.6,0}
\definecolor{codegray}{rgb}{0.5,0.5,0.5}
\definecolor{codepurple}{rgb}{0.58,0,0.82}
\definecolor{backcolour}{rgb}{0.95,0.95,0.92}

\lstdefinestyle{mystyle}{
    backgroundcolor=\color{backcolour},
    commentstyle=\color{codegreen},
    keywordstyle=\color{magenta},
    numberstyle=\tiny\color{codegray},
    stringstyle=\color{codepurple},
    basicstyle=\ttfamily\footnotesize,
    breakatwhitespace=false,
    breaklines=true,
    captionpos=b,
    keepspaces=true,
    numbers=left,
    numbersep=5pt,
    showspaces=false,
    showstringspaces=false,
    showtabs=false,
    tabsize=2
}
\lstset{style=mystyle}

% Algorithms
\usepackage{algorithm}
\usepackage{algpseudocode}

% References and links
\usepackage[hidelinks]{hyperref}
\usepackage{cleveref}

% Bibliography
\usepackage[style=numeric,sorting=none,backend=biber]{biblatex}
\addbibresource{references.bib}

% Misc
\usepackage{enumitem}
\usepackage{parskip}
\usepackage{fancyhdr}

% Header/Footer style
\pagestyle{fancy}
\fancyhf{}
\fancyhead[L]{\leftmark}
\fancyhead[R]{\thepage}
\renewcommand{\headrulewidth}{0.4pt}

% ============================================================================
% DOCUMENT METADATA
% ============================================================================

\title{Reinforcement Learning for Optimizing\\Retrieval-Augmented Generation:\\A Self-Improving RAG Agent for Academic Question Answering}
\author{Kristian Rahnev}
\date{2026}

% ============================================================================
% DOCUMENT
% ============================================================================

\begin{document}

% ----------------------------------------------------------------------------
% FRONT MATTER
% ----------------------------------------------------------------------------

% Title page
\begin{titlepage}
    \centering
    \vspace*{2cm}

    {\Large\textbf{[University Name]}}\\[0.5cm]
    {\large Department of Computer Science}\\[2cm]

    {\LARGE\textbf{Reinforcement Learning for Optimizing\\Retrieval-Augmented Generation}}\\[0.5cm]
    {\Large A Self-Improving RAG Agent for Academic Question Answering}\\[2cm]

    {\large BSc Computer Science\\Final Year Dissertation}\\[2cm]

    {\Large\textbf{[Your Name]}}\\[0.5cm]
    {\large Student ID: [Your ID]}\\[2cm]

    {\large Supervisor: [Supervisor Name]}\\[2cm]

    {\large January 2026}

    \vfill
\end{titlepage}

% Abstract
\chapter*{Abstract}
\addcontentsline{toc}{chapter}{Abstract}

Retrieval-Augmented Generation (RAG) systems enhance Large Language Model (LLM) capabilities by grounding responses in external knowledge. However, traditional RAG systems use fixed retrieval strategies, retrieving documents for every query regardless of necessity. This dissertation presents an approach to optimizing RAG systems using Reinforcement Learning (RL), training a policy network to make intelligent retrieval decisions.

We developed an RL-enhanced RAG pipeline with a neural policy network trained using the REINFORCE algorithm. The system learns when retrieval is beneficial versus when the LLM's parametric knowledge suffices. We identified and solved the ``lazy agent'' problem---where the policy learns to always skip retrieval---through reward shaping, curriculum learning, entropy regularization, and temperature-based sampling.

Experiments on a custom academic question-answering dataset (492 training samples from 2,000 arXiv papers) demonstrate that the trained policy achieves 33.87\% F1 score, matching the always-retrieve baseline (34.08\% F1) while correctly learning that academic domain questions require retrieval. The policy outperforms the no-retrieval baseline (18.52\% F1) by 83\%.

This work contributes: (1) a modular RL-RAG training framework, (2) solutions to the lazy agent problem in retrieval policy learning, and (3) empirical evidence that RL can discover optimal retrieval strategies for domain-specific question answering.

\textbf{Keywords:} Retrieval-Augmented Generation, Reinforcement Learning, Large Language Models, Question Answering, Policy Gradient Methods

% Acknowledgements
\chapter*{Acknowledgements}
\addcontentsline{toc}{chapter}{Acknowledgements}

[ I would like to thank my supervisor, Dr. Sama Aleshaiker, for their guidance                                                                                                    
and feedback throughout this project. ]

% Table of contents
\tableofcontents
\listoffigures
\listoftables

% ----------------------------------------------------------------------------
% MAIN MATTER
% ----------------------------------------------------------------------------

% ============================================================================
% Chapter 1: Introduction
% ============================================================================

\chapter{Introduction}
\label{ch:introduction}

\section{Motivation}

Large Language Models (LLMs) have demonstrated remarkable capabilities in natural language understanding and generation. However, they suffer from fundamental limitations: knowledge cutoffs, hallucination of facts, and inability to access domain-specific or up-to-date information. Retrieval-Augmented Generation (RAG) addresses these limitations by augmenting LLM responses with relevant documents retrieved from external knowledge bases.

Traditional RAG systems employ fixed retrieval strategies---retrieving documents for every query regardless of whether external knowledge is actually needed. This approach is inefficient: simple factual questions within the LLM's training knowledge trigger unnecessary retrieval operations, increasing latency and computational costs. Conversely, complex multi-hop reasoning questions may require multiple retrieval iterations that fixed strategies cannot provide.

This dissertation investigates whether Reinforcement Learning (RL) can optimize RAG systems by learning \emph{when} to retrieve, creating a self-improving agent that adapts its retrieval strategy based on query characteristics and domain requirements.

\section{Research Questions}

This dissertation addresses the following research questions:

\begin{enumerate}
    \item \textbf{RQ1:} Can an RL-trained policy learn to make effective retrieval decisions in a RAG system?

    \item \textbf{RQ2:} What reward function design enables stable training without degenerate behaviors (e.g., never retrieving)?

    \item \textbf{RQ3:} How does an RL-optimized retrieval policy compare to fixed retrieval baselines on domain-specific question answering?
\end{enumerate}

\section{Contributions}

This dissertation makes the following contributions:

\begin{enumerate}
    \item \textbf{RL-RAG Framework:} A modular pipeline integrating neural policy networks with RAG components (dense retrieval, query rewriting, LLM generation) trained using policy gradient methods.

    \item \textbf{Lazy Agent Solutions:} Identification and resolution of the ``lazy agent'' problem---where policies learn to always skip retrieval---through five complementary techniques: reward shaping, curriculum learning, entropy regularization, temperature-based sampling, and question difficulty features.

    \item \textbf{Custom Academic Dataset:} A curated question-answering dataset of 492 high-quality QA pairs generated from 2,000 arXiv papers, with multi-stage quality filtering achieving $<5\%$ false positive rate.

    \item \textbf{Comprehensive Baseline Study:} Systematic evaluation of 8+ RAG configurations comparing sparse (BM25) vs. dense (E5) retrieval and local (Ollama) vs. cloud (GPT-4o-mini) generation.

    \item \textbf{Empirical Validation:} Demonstration that RL can discover optimal retrieval strategies, with the trained policy matching baseline performance (33.87\% vs.\ 34.08\% F1) while learning domain-appropriate behavior.
\end{enumerate}

\section{Dissertation Structure}

The remainder of this dissertation is organized as follows:

\begin{description}
    \item[Chapter 2: Background] Reviews foundational concepts in RAG systems, reinforcement learning, and related work on adaptive retrieval.

    \item[Chapter 3: Methodology] Presents the RL-RAG framework, including the policy network architecture, reward function design, and training algorithms.

    \item[Chapter 4: Implementation] Describes the technical implementation, including the FlashRAG integration, dataset preparation, and experimental infrastructure.

    \item[Chapter 5: Experiments] Reports experimental results on baseline comparisons, RL training, and the lazy agent problem resolution.

    \item[Chapter 6: Discussion] Analyzes findings, limitations, and implications for RAG system optimization.

    \item[Chapter 7: Conclusion] Summarizes contributions and outlines future research directions.
\end{description}

% ============================================================================
% Chapter 2: Background
% ============================================================================

\chapter{Background}
\label{ch:background}

This chapter reviews the foundational concepts underlying this dissertation: Retrieval-Augmented Generation, reinforcement learning for language models, and existing approaches to adaptive retrieval.

\section{Retrieval-Augmented Generation}

\subsection{Core Architecture}

Retrieval-Augmented Generation (RAG) systems combine information retrieval with neural text generation. The standard RAG pipeline consists of three components:

\begin{enumerate}
    \item \textbf{Retriever:} Given a query $q$, retrieves relevant documents $D = \{d_1, d_2, \ldots, d_k\}$ from a corpus $\mathcal{C}$.

    \item \textbf{Augmentation:} Constructs a prompt combining the query and retrieved documents.

    \item \textbf{Generator:} A language model generates an answer conditioned on the augmented prompt.
\end{enumerate}

Formally, the generation probability is:
\begin{equation}
    P(a | q) = \sum_{d \in D} P(a | q, d) \cdot P(d | q)
\end{equation}

where $P(d | q)$ is the retrieval probability and $P(a | q, d)$ is the generation probability given the retrieved context.

\subsection{Retrieval Methods}

Two primary retrieval paradigms exist:

\textbf{Sparse Retrieval (BM25):} Uses term frequency statistics with the Okapi BM25 scoring function:
\begin{equation}
    \text{BM25}(q, d) = \sum_{t \in q} \text{IDF}(t) \cdot \frac{f(t, d) \cdot (k_1 + 1)}{f(t, d) + k_1 \cdot (1 - b + b \cdot \frac{|d|}{\text{avgdl}})}
\end{equation}

where $f(t, d)$ is term frequency, $|d|$ is document length, and $k_1$, $b$ are tuning parameters.

\textbf{Dense Retrieval:} Encodes queries and documents into dense vector representations using neural encoders. Relevance is computed via similarity (e.g., dot product or cosine):
\begin{equation}
    \text{sim}(q, d) = \mathbf{e}_q^\top \mathbf{e}_d
\end{equation}

where $\mathbf{e}_q$ and $\mathbf{e}_d$ are embeddings from models like E5~\cite{e5} or DPR~\cite{dpr}.

\subsection{FlashRAG Toolkit}

FlashRAG~\cite{flashrag} provides a modular framework implementing 23+ RAG methods. Its architecture includes:

\begin{itemize}
    \item \textbf{Retrievers:} Dense (E5, BGE, DPR), sparse (BM25), and hybrid methods
    \item \textbf{Refiners:} Extractive selection, abstractive compression, LLMLingua
    \item \textbf{Generators:} HuggingFace, vLLM, OpenAI API, Ollama
    \item \textbf{Pipelines:} Sequential, conditional (judger-based), branching, and loop-based
\end{itemize}

\section{Reinforcement Learning Fundamentals}

\subsection{Markov Decision Process}

RL problems are formalized as Markov Decision Processes (MDPs) defined by the tuple $(\mathcal{S}, \mathcal{A}, P, R, \gamma)$:

\begin{itemize}
    \item $\mathcal{S}$: State space
    \item $\mathcal{A}$: Action space
    \item $P(s' | s, a)$: Transition probability
    \item $R(s, a)$: Reward function
    \item $\gamma$: Discount factor
\end{itemize}

The objective is to learn a policy $\pi(a | s)$ maximizing expected cumulative reward:
\begin{equation}
    J(\pi) = \mathbb{E}_{\tau \sim \pi} \left[ \sum_{t=0}^{T} \gamma^t R(s_t, a_t) \right]
\end{equation}

\subsection{Policy Gradient Methods}

Policy gradient methods directly optimize the policy by computing gradients of the expected reward:
\begin{equation}
    \nabla_\theta J(\theta) = \mathbb{E}_{\tau \sim \pi_\theta} \left[ \sum_{t=0}^{T} \nabla_\theta \log \pi_\theta(a_t | s_t) \cdot G_t \right]
\end{equation}

where $G_t = \sum_{k=t}^{T} \gamma^{k-t} R_k$ is the return from time $t$.

\textbf{REINFORCE}~\cite{williams1992} is a Monte Carlo policy gradient algorithm that estimates the gradient using sampled trajectories:
\begin{equation}
    \nabla_\theta J(\theta) \approx \frac{1}{N} \sum_{i=1}^{N} \sum_{t=0}^{T} \nabla_\theta \log \pi_\theta(a_t^{(i)} | s_t^{(i)}) \cdot (G_t^{(i)} - b)
\end{equation}

where $b$ is a baseline to reduce variance.

\subsection{Entropy Regularization}

To encourage exploration, an entropy bonus is often added to the objective:
\begin{equation}
    J_{\text{entropy}}(\theta) = J(\theta) + \beta H(\pi_\theta)
\end{equation}

where $H(\pi_\theta) = -\sum_a \pi_\theta(a|s) \log \pi_\theta(a|s)$ is the policy entropy and $\beta$ controls the exploration-exploitation trade-off.

\section{RL for Retrieval Optimization}

\subsection{Adaptive Retrieval Decisions}

Several works have explored learning when to retrieve:

\textbf{Self-RAG}~\cite{selfrag}: Trains the LLM itself to output special tokens indicating whether retrieval is needed, achieving adaptive retrieval without a separate policy.

\textbf{Adaptive-RAG}~\cite{adaptiverag}: Uses query complexity classification to route queries to different retrieval strategies (no retrieval, single-hop, multi-hop).

\textbf{FLARE}~\cite{flare}: Monitors generation confidence and triggers retrieval when the model is uncertain, using token probabilities as the decision signal.

\subsection{RL for RAG Optimization}

\textbf{Agent Lightning}~\cite{agentlightning}: Provides a framework for training RAG agents using RL, modeling the pipeline as an MDP and optimizing with policy gradient methods.

\textbf{MMOA-RAG}~\cite{mmoarag}: Models RAG as a cooperative multi-agent RL problem, jointly optimizing query rewriting, document selection, and generation.

\textbf{MaFeRw}~\cite{maferw}: Uses multi-aspect dense rewards (retrieval metrics, ROUGE scores) to stabilize RL training for query rewriting.

\subsection{Cost-Aware Retrieval}

Kulkarni et al.~\cite{kulkarni} propose an RL policy with a binary action space [\texttt{FETCH}] vs. [\texttt{NO\_FETCH}], optimizing for answer quality minus retrieval cost. This directly inspired our approach.

\section{Evaluation Metrics}

\subsection{Answer Quality}

\textbf{Exact Match (EM):} Binary metric indicating whether the prediction exactly matches a ground truth answer after normalization:
\begin{equation}
    \text{EM} = \mathbf{1}[\text{normalize}(\hat{a}) = \text{normalize}(a^*)]
\end{equation}

\textbf{F1 Score:} Token-level F1 between prediction and ground truth:
\begin{equation}
    \text{F1} = \frac{2 \cdot \text{Precision} \cdot \text{Recall}}{\text{Precision} + \text{Recall}}
\end{equation}

where precision and recall are computed over token overlap.

\subsection{Retrieval Quality}

\textbf{Recall@k:} Proportion of queries where at least one relevant document appears in the top-$k$ retrieved:
\begin{equation}
    \text{Recall@k} = \frac{1}{|Q|} \sum_{q \in Q} \mathbf{1}[\exists d \in D_k(q) : d \text{ is relevant}]
\end{equation}

We use token-overlap matching ($\geq 50\%$ overlap threshold) rather than exact substring matching to handle paraphrasing in academic text.

\section{Summary}

This chapter established the theoretical foundations for this dissertation. RAG systems augment LLMs with retrieved knowledge, but fixed retrieval strategies are suboptimal. RL provides a framework for learning adaptive policies, with policy gradient methods enabling optimization of discrete retrieval decisions. Prior work has explored adaptive retrieval, but the problem of training stable policies that avoid degenerate behaviors remains open.

% ============================================================================
% Chapter 3: Methodology
% ============================================================================

\chapter{Methodology}
\label{ch:methodology}

This chapter presents the methodology for training an RL-optimized RAG system. We formalize the problem as an MDP, describe the policy network architecture, and detail the reward function design including solutions to the lazy agent problem.

\section{Problem Formulation}

\subsection{RAG as a Markov Decision Process}

We model the RAG question-answering task as an episodic MDP where each episode corresponds to answering a single question.

\textbf{State Space} $\mathcal{S}$: The state $s$ is the question $q$ represented as a dense embedding from a sentence encoder.

\textbf{Action Space} $\mathcal{A}$: Binary action space:
\begin{equation}
    \mathcal{A} = \{\texttt{RETRIEVE}, \texttt{GENERATE\_DIRECTLY}\}
\end{equation}

\textbf{Transition Dynamics}: Deterministic---after selecting an action, the system retrieves (or not) and generates an answer.

\textbf{Reward}: Based on answer quality (F1 score) with retrieval cost penalties and efficiency bonuses (detailed in Section~\ref{sec:reward}).

\textbf{Episode}: Single-step---the agent observes a question, takes one action, receives reward, and the episode terminates.

\subsection{Objective}

The objective is to learn a policy $\pi_\theta(a | s)$ that maximizes expected reward:
\begin{equation}
    \max_\theta \mathbb{E}_{q \sim \mathcal{D}, a \sim \pi_\theta} \left[ R(q, a, \hat{a}, a^*) \right]
\end{equation}

where $\hat{a}$ is the generated answer and $a^*$ is the ground truth.

\section{Policy Network Architecture}

\subsection{Question Encoder}

Questions are encoded using a pre-trained sentence transformer (E5-base-v2):
\begin{equation}
    \mathbf{h} = \text{Encoder}(q) \in \mathbb{R}^{768}
\end{equation}

\subsection{Policy Network}

The policy network is a multi-layer perceptron that maps question embeddings to retrieval probabilities:

\begin{equation}
    \pi_\theta(a = \texttt{RETRIEVE} | q) = \sigma(\text{MLP}(\mathbf{h}))
\end{equation}

Architecture:
\begin{align}
    \mathbf{z}_1 &= \text{ReLU}(\mathbf{W}_1 \mathbf{h} + \mathbf{b}_1) & \mathbf{W}_1 \in \mathbb{R}^{256 \times 768} \\
    \mathbf{z}_2 &= \text{ReLU}(\mathbf{W}_2 \mathbf{z}_1 + \mathbf{b}_2) & \mathbf{W}_2 \in \mathbb{R}^{64 \times 256} \\
    p &= \sigma(\mathbf{w}_3^\top \mathbf{z}_2 + b_3) & \mathbf{w}_3 \in \mathbb{R}^{64}
\end{align}

where $\sigma$ is the sigmoid function and $p$ is the probability of retrieving.

\subsection{Action Selection}

During training, actions are sampled stochastically with epsilon-greedy exploration:
\begin{equation}
    a = \begin{cases}
        \text{Uniform}(\mathcal{A}) & \text{with probability } \epsilon \\
        \text{Bernoulli}(p) & \text{otherwise}
    \end{cases}
\end{equation}

During evaluation, we use temperature-scaled sampling (Section~\ref{sec:temperature}) rather than deterministic argmax.

\section{Reward Function Design}
\label{sec:reward}

\subsection{Base Reward Components}

The reward function balances answer quality with retrieval efficiency:

\begin{equation}
    R = R_{\text{quality}} + R_{\text{retrieval}} + R_{\text{format}}
\end{equation}

\textbf{Quality Reward} $R_{\text{quality}}$: Token-level F1 score between prediction and ground truth:
\begin{equation}
    R_{\text{quality}} = \text{F1}(\hat{a}, a^*) \in [0, 1]
\end{equation}

\textbf{Retrieval Reward} $R_{\text{retrieval}}$: Cost for retrieval, bonus for correct answers without retrieval:
\begin{equation}
    R_{\text{retrieval}} = \begin{cases}
        -c_{\text{retr}} & \text{if retrieved} \\
        +b_{\text{eff}} & \text{if not retrieved and F1} \geq \tau_{\text{correct}} \\
        -p_{\text{lazy}} & \text{if not retrieved and F1} < \tau_{\text{bad}} \\
        0 & \text{otherwise}
    \end{cases}
\end{equation}

\textbf{Format Reward} $R_{\text{format}}$: Small bonus for well-formed answers:
\begin{equation}
    R_{\text{format}} = \begin{cases}
        +0.05 & \text{if } 0 < |\hat{a}| < 500 \\
        0 & \text{otherwise}
    \end{cases}
\end{equation}

\subsection{Hyperparameters}

\begin{table}[H]
    \centering
    \caption{Reward function hyperparameters}
    \label{tab:reward-params}
    \begin{tabular}{lcc}
        \toprule
        Parameter & Symbol & Value \\
        \midrule
        Retrieval cost & $c_{\text{retr}}$ & 0.1 \\
        Efficiency bonus & $b_{\text{eff}}$ & 0.1 \\
        Lazy agent penalty & $p_{\text{lazy}}$ & 0.3 \\
        Correct threshold & $\tau_{\text{correct}}$ & 0.5 \\
        Bad answer threshold & $\tau_{\text{bad}}$ & 0.3 \\
        Format bonus & - & 0.05 \\
        \bottomrule
    \end{tabular}
\end{table}

\section{The Lazy Agent Problem}

\subsection{Problem Description}

During initial experiments, we observed that the policy consistently learned to \emph{never} retrieve during evaluation, achieving 0\% retrieval rate. We term this the ``lazy agent'' problem.

\textbf{Symptoms:}
\begin{itemize}
    \item Training shows exploration (26--72\% retrieval rate due to epsilon-greedy)
    \item Evaluation uses deterministic policy $\rightarrow$ 0\% retrieval
    \item Validation F1 stuck at $\sim$0.14 (LLM parametric knowledge only)
\end{itemize}

\textbf{Root Cause:} The reward structure made ``no retrieval'' appear optimal:
\begin{itemize}
    \item F1 without retrieval ($\sim$0.14) with no cost $>$ F1 with retrieval ($\sim$0.31) minus cost (0.1)
    \item The efficiency bonus (+0.1) further incentivized skipping retrieval
    \item Deterministic evaluation amplified this bias
\end{itemize}

\subsection{Solution 1: Reward Shaping}
\label{sec:reward-shaping}

The critical fix was introducing a penalty for bad answers without retrieval that \emph{exceeds} the retrieval cost:

\begin{equation}
    p_{\text{lazy}} > c_{\text{retr}}
\end{equation}

With $p_{\text{lazy}} = 0.3$ and $c_{\text{retr}} = 0.1$, the expected value of retrieving exceeds not retrieving when the LLM alone produces bad answers (F1 $< 0.3$).

\textbf{Intuition:} The agent learns that the ``safe'' choice is to retrieve, because the penalty for being wrong without retrieval outweighs the cost of retrieving.

\subsection{Solution 2: Curriculum Learning}
\label{sec:curriculum}

We implemented phased training with decreasing minimum retrieval rates:

\begin{table}[H]
    \centering
    \caption{Curriculum learning phases}
    \label{tab:curriculum}
    \begin{tabular}{cccc}
        \toprule
        Phase & Epochs & Min Retrieval Rate & Purpose \\
        \midrule
        1 & 1--3 & 80\% & Learn retrieval value \\
        2 & 4--6 & 40\% & Gradual autonomy \\
        3 & 7+ & 0\% & Full policy control \\
        \bottomrule
    \end{tabular}
\end{table}

During curriculum phases, random samples are forced to retrieve regardless of policy output:
\begin{equation}
    a = \begin{cases}
        \texttt{RETRIEVE} & \text{with probability } r_{\min} \\
        \text{Policy}(q) & \text{otherwise}
    \end{cases}
\end{equation}

\subsection{Solution 3: Entropy Regularization}
\label{sec:entropy}

We add an entropy bonus to the policy loss to encourage exploration:
\begin{equation}
    \mathcal{L} = -\log \pi_\theta(a | s) \cdot (R - b) - \beta H(\pi_\theta(s))
\end{equation}

where:
\begin{equation}
    H(\pi_\theta(s)) = -p \log p - (1-p) \log(1-p)
\end{equation}

We use $\beta = 0.01$.

\subsection{Solution 4: Temperature-Based Sampling}
\label{sec:temperature}

Instead of deterministic argmax during evaluation, we use temperature-scaled sampling:
\begin{equation}
    p_{\text{temp}} = \sigma\left(\frac{\text{logit}(p)}{T}\right)
\end{equation}

where $\text{logit}(p) = \log(p / (1-p))$ and $T = 0.7$ is the temperature.

This prevents the policy from collapsing to deterministic behavior during evaluation.

\subsection{Solution 5: Question Difficulty Features}

Optionally, we augment the state with question difficulty features:
\begin{itemize}
    \item Question length (characters, words)
    \item Presence of named entities
    \item Question type (what, who, how, why, etc.)
    \item Complexity indicators
\end{itemize}

These features help the policy distinguish questions that can be answered directly from those requiring retrieval.

\section{Training Algorithm}

\subsection{REINFORCE with Baseline}

We use the REINFORCE algorithm with a learned baseline for variance reduction:

\begin{algorithm}[H]
\caption{RL-RAG Training}
\label{alg:training}
\begin{algorithmic}[1]
\Require Training data $\mathcal{D}$, policy $\pi_\theta$, baseline $b$
\For{epoch $= 1$ to $E$}
    \State Set curriculum parameters based on epoch
    \State $\epsilon \gets \epsilon_0 \cdot \gamma^{\text{epoch}}$ \Comment{Decay exploration}
    \For{batch in $\mathcal{D}$}
        \For{question $q$ in batch}
            \State $\mathbf{h} \gets \text{Encoder}(q)$
            \State $p \gets \pi_\theta(\mathbf{h})$
            \State Sample action $a$ (with epsilon-greedy and curriculum)
            \If{$a = \texttt{RETRIEVE}$}
                \State $D \gets \text{Retrieve}(q)$
                \State $\hat{a} \gets \text{Generate}(q, D)$
            \Else
                \State $\hat{a} \gets \text{Generate}(q)$
            \EndIf
            \State Compute reward $R$ and advantage $A = R - b$
            \State Store $(p, a, A, H)$ for update
        \EndFor
        \If{update step}
            \State $\mathcal{L} \gets -\mathbb{E}[\log \pi_\theta(a|s) \cdot A + \beta H]$
            \State $\theta \gets \theta - \alpha \nabla_\theta \mathcal{L}$
        \EndIf
    \EndFor
    \State Evaluate on validation set
\EndFor
\end{algorithmic}
\end{algorithm}

\subsection{Hyperparameters}

\begin{table}[H]
    \centering
    \caption{Training hyperparameters}
    \label{tab:training-params}
    \begin{tabular}{lc}
        \toprule
        Parameter & Value \\
        \midrule
        Epochs & 10 \\
        Initial epsilon & 0.5 \\
        Epsilon decay & 0.7 per epoch \\
        Learning rate & $10^{-3}$ \\
        Update frequency & Every 5 samples \\
        Entropy coefficient $\beta$ & 0.01 \\
        Evaluation temperature & 0.7 \\
        Curriculum phases & 3 \\
        \bottomrule
    \end{tabular}
\end{table}

\section{Summary}

This chapter presented our methodology for training RL-optimized RAG systems. We formalized the problem as an MDP with a binary action space, designed a reward function balancing quality and efficiency, and developed five solutions to the lazy agent problem. The training algorithm uses REINFORCE with curriculum learning and entropy regularization.

% ============================================================================
% Chapter 4: Implementation
% ============================================================================

\chapter{Implementation}
\label{ch:implementation}

This chapter describes the technical implementation of the RL-RAG system, including dataset preparation, system architecture, and experimental infrastructure.

\section{Dataset Preparation}

\subsection{Source Documents}

We collected 2,000 academic papers from arXiv covering diverse scientific domains. The corpus was split into:
\begin{itemize}
    \item \textbf{Training:} 1,600 PDFs
    \item \textbf{Test:} 400 PDFs
\end{itemize}

\subsection{QA Pair Generation}

Question-answer pairs were generated using a multi-stage pipeline:

\textbf{Step 1: Text Extraction and Chunking}
\begin{itemize}
    \item PDFs processed with PyPDFLoader
    \item Text chunked using RecursiveCharacterTextSplitter (chunk size: 1,500 characters, overlap: 200)
    \item Random sampling with fixed seed (42) for reproducibility
\end{itemize}

\textbf{Step 2: QA Generation}
\begin{itemize}
    \item Model: Llama 3.1 8B via Ollama (local)
    \item Target: 1,000 training pairs, 200 test pairs
    \item Prompt engineering for educational QA pairs with JSON output
\end{itemize}

\textbf{Step 3: Quality Assessment}

Initial assessment using Claude 3 Haiku revealed quality issues (15--20\% false positive rate). We switched to GPT-4o-mini with multi-stage filtering:

\begin{enumerate}
    \item \textbf{Rule-based pre-filtering:} Automated detection of trivial questions, incomplete answers, math notation errors (reduces API calls by 25--30\%)

    \item \textbf{LLM assessment:} GPT-4o-mini scores each pair on four criteria (1--10 scale):
    \begin{itemize}
        \item Question quality
        \item Answer accuracy
        \item Relevance
        \item Educational value
    \end{itemize}

    \item \textbf{Threshold-based categorization:}
    \begin{itemize}
        \item High quality ($\geq 8.5$): Auto-approved
        \item Borderline (7.5--8.5): Manual review
        \item Rejected ($< 7.5$): Excluded
    \end{itemize}

    \item \textbf{Borderline promotion:} Pairs with score $\geq 8.0$ promoted to filtered dataset
\end{enumerate}

\subsection{Final Dataset Statistics}

\begin{table}[H]
    \centering
    \caption{Dataset statistics after quality filtering}
    \label{tab:dataset-stats}
    \begin{tabular}{lcc}
        \toprule
        Metric & Training & Test \\
        \midrule
        QA pairs & 492 & 87 \\
        Average score & 8.35/10 & 8.30/10 \\
        Score range & 8.00--9.50 & 8.00--9.50 \\
        False positive rate & $<5\%$ & $<5\%$ \\
        \bottomrule
    \end{tabular}
\end{table}

\subsection{Corpus Preparation}

Documents were processed for retrieval:

\begin{enumerate}
    \item \textbf{Text extraction:} PyPDFLoader with Unicode cleaning
    \item \textbf{Chunking:} Sentence-based, 512 words per chunk
    \item \textbf{Result:} 41,717 chunks from training corpus
\end{enumerate}

\begin{table}[H]
    \centering
    \caption{Corpus statistics}
    \label{tab:corpus-stats}
    \begin{tabular}{lcc}
        \toprule
        Corpus & Documents & Chunks \\
        \midrule
        Training & 1,600 & 34,040 \\
        Test & 398 & 7,679 \\
        Combined & 1,998 & 41,717 \\
        \bottomrule
    \end{tabular}
\end{table}

\section{System Architecture}

\subsection{Component Overview}

The system consists of modular components organized in the \texttt{src/agents/} directory:

\begin{figure}[H]
    \centering
    \begin{verbatim}
    src/agents/
    |-- enhanced_pipeline.py    # Main RL-RAG pipeline
    |-- flashrag_components.py  # Retriever/Generator wrappers
    |-- reward.py               # Reward calculation
    |-- dataset.py              # Data loading utilities
    +-- rl_rag_agent.py         # Agent Lightning integration
    \end{verbatim}
    \caption{Source code organization}
    \label{fig:source-org}
\end{figure}

\subsection{Dense Retriever}

We implemented a custom dense retriever wrapping E5-base-v2:

\begin{lstlisting}[language=Python, caption=Dense retriever implementation]
class DenseRetrieverWrapper:
    def __init__(self, model_name="intfloat/e5-base-v2"):
        self.model = SentenceTransformer(model_name)
        self.index = faiss.read_index(index_path)
        self.corpus = load_corpus(corpus_path)

    def search(self, query: str, topk: int = 5):
        # Add retrieval instruction for E5
        query = f"query: {query}"
        embedding = self.model.encode([query])
        scores, indices = self.index.search(embedding, topk)
        return [self.corpus[i] for i in indices[0]]
\end{lstlisting}

\textbf{Index:} FAISS Flat index with 768-dimensional vectors (128 MB, 41,717 vectors).

\subsection{Generator}

The generator wraps OpenAI's GPT-4o-mini API with retry logic and cost tracking:

\begin{lstlisting}[language=Python, caption=Generator with cost tracking]
class GeneratorWrapper:
    PRICING = {
        "gpt-4o-mini": {
            "input": 0.15 / 1_000_000,
            "output": 0.60 / 1_000_000
        }
    }

    def generate(self, query: str, context: List[str] = None):
        prompt = self._build_prompt(query, context)
        response = self.client.chat.completions.create(
            model=self.model,
            messages=[{"role": "user", "content": prompt}],
            max_tokens=256,
            temperature=0.0
        )
        self._track_usage(response.usage)
        return response.choices[0].message.content
\end{lstlisting}

\subsection{Policy Network}

The policy network is implemented in PyTorch:

\begin{lstlisting}[language=Python, caption=Policy network architecture]
class RetrievalPolicyNetwork(nn.Module):
    def __init__(self, input_dim=768, hidden_dim=256):
        super().__init__()
        self.network = nn.Sequential(
            nn.Linear(input_dim, hidden_dim),
            nn.ReLU(),
            nn.Linear(hidden_dim, 64),
            nn.ReLU(),
            nn.Linear(64, 1),
            nn.Sigmoid()
        )

    def forward(self, x):
        return self.network(x)

    def get_action(self, x, deterministic=False, temperature=1.0):
        prob = self.forward(x)
        if not deterministic and temperature != 1.0:
            logit = torch.log(prob / (1 - prob + 1e-8))
            prob = torch.sigmoid(logit / temperature)
        action = prob > 0.5 if deterministic else torch.bernoulli(prob)
        log_prob = torch.log(prob if action else 1 - prob)
        entropy = -(prob * torch.log(prob + 1e-8) +
                   (1-prob) * torch.log(1-prob + 1e-8))
        return action.item(), log_prob, entropy
\end{lstlisting}

\subsection{Reward Calculator}

The reward calculator implements the reward function from Section~\ref{sec:reward}:

\begin{lstlisting}[language=Python, caption=Reward calculation with lazy agent penalty]
class RAGRewardCalculator:
    def compute_reward(self, prediction, ground_truths,
                       did_retrieve, num_retrievals=1):
        f1_score = compute_f1(prediction, ground_truths)
        is_correct = f1_score >= self.f1_threshold_for_correct
        is_bad = f1_score < self.f1_threshold_for_bad

        reward = f1_score  # Base quality reward

        if did_retrieve:
            reward -= self.retrieval_cost * num_retrievals
        else:
            if is_correct:
                reward += self.correct_no_retrieval_bonus
            elif is_bad:
                # Critical fix: penalize bad no-retrieval
                reward -= self.wrong_no_retrieval_penalty

        return reward, {"f1": f1_score, "did_retrieve": did_retrieve}
\end{lstlisting}

\section{Training Infrastructure}

\subsection{Training Script}

The main training script (\texttt{train\_enhanced\_rag.py}) supports:

\begin{itemize}
    \item Command-line hyperparameter configuration
    \item Weights \& Biases logging
    \item Curriculum learning phases
    \item Model checkpointing
    \item Cost tracking
\end{itemize}

\begin{lstlisting}[language=bash, caption=Training command]
python experiments/scripts/rl/train_enhanced_rag.py \
    --dataset custom \
    --samples 492 \
    --epochs 10 \
    --retrieval-cost 0.1 \
    --wrong-no-retrieval-penalty 0.3 \
    --entropy-coef 0.01 \
    --wandb
\end{lstlisting}

\subsection{Evaluation Script}

The evaluation script compares trained policies against baselines:

\begin{lstlisting}[language=bash, caption=Evaluation command]
python experiments/scripts/rl/evaluate_rl_agent.py \
    --mode trained \
    --checkpoint path/to/best_model.pt \
    --samples 87
\end{lstlisting}

\subsection{Experiment Tracking}

We use Weights \& Biases for experiment tracking with comprehensive logging:

\begin{itemize}
    \item Per-sample metrics: F1, EM, reward, retrieval decision
    \item Epoch summaries: average F1, retrieval rate, entropy, loss
    \item Training curves: learning dynamics over epochs
    \item Hyperparameter tracking: full configuration logged
\end{itemize}

\section{Technical Challenges and Solutions}

\subsection{FlashRAG Pipeline Issues}

FlashRAG's built-in pipelines caused segmentation faults due to FAISS/multiprocessing/MPS interactions. We bypassed this by implementing custom wrappers that directly call retrieval and generation components.

\subsection{OpenMP Conflict}

macOS systems experienced OpenMP library conflicts. We added environment variable fixes:

\begin{lstlisting}[language=Python]
os.environ['KMP_DUPLICATE_LIB_OK'] = 'TRUE'
os.environ['TOKENIZERS_PARALLELISM'] = 'false'
\end{lstlisting}

\subsection{API Rate Limiting}

GPT-4o-mini has rate limits (10,000 requests per day). We implemented:
\begin{itemize}
    \item Exponential backoff retry logic
    \item Request tracking and cost estimation
    \item Option to disable query rewriter to halve API calls
\end{itemize}

\section{Hardware and Software}

\begin{table}[H]
    \centering
    \caption{Experimental environment}
    \label{tab:environment}
    \begin{tabular}{ll}
        \toprule
        Component & Specification \\
        \midrule
        Hardware & Apple M3 Pro, 18GB RAM \\
        OS & macOS 14.6 \\
        Python & 3.10.19 \\
        PyTorch & 2.x (MPS backend) \\
        Sentence Transformers & 2.x \\
        FAISS & 1.8.0 (CPU) \\
        OpenAI API & gpt-4o-mini \\
        \bottomrule
    \end{tabular}
\end{table}

\section{Summary}

This chapter described the implementation of our RL-RAG system. We created a high-quality academic QA dataset through multi-stage filtering, implemented modular RAG components with cost tracking, and built training infrastructure with comprehensive experiment tracking. Technical challenges including FlashRAG compatibility and API rate limiting were addressed through custom implementations and careful engineering.

% ============================================================================
% Chapter 5: Experiments
% ============================================================================

\chapter{Experiments}
\label{ch:experiments}

This chapter presents experimental results in three parts: baseline evaluation, RL training with lazy agent analysis, and final performance comparison.

\section{Baseline Experiments}

\subsection{Experimental Setup}

We evaluated multiple RAG configurations to establish baselines:

\begin{table}[H]
    \centering
    \caption{Baseline configurations evaluated}
    \label{tab:baseline-configs}
    \begin{tabular}{llll}
        \toprule
        Method & Retrieval & Generator & Description \\
        \midrule
        Naive RAG & BM25 & Ollama & Sparse retrieval + local LLM \\
        Naive RAG & BM25 & GPT-4o-mini & Sparse retrieval + cloud LLM \\
        Dense RAG & E5 + FAISS & GPT-4o-mini & Semantic search + cloud LLM \\
        Iter-RetGen & BM25 & Ollama & Iterative retrieval \\
        Adaptive-RAG & BM25 & Ollama & Query complexity routing \\
        \bottomrule
    \end{tabular}
\end{table}

\subsection{Baseline Results}

\begin{table}[H]
    \centering
    \caption{Complete baseline performance on custom dataset (87 test samples)}
    \label{tab:baseline-results}
    \begin{tabular}{lccc}
        \toprule
        Method & EM (\%) & F1 (\%) & Recall@k (\%) \\
        \midrule
        Dense E5 (topk=10) + GPT-4o-mini & 3.45 & \textbf{31.10} & \textbf{87.36} \\
        Dense E5 (topk=5) + GPT-4o-mini & 3.45 & 30.82 & 83.91 \\
        Dense E5 (topk=3) + GPT-4o-mini & \textbf{4.60} & 30.08 & 78.16 \\
        BM25 + GPT-4o-mini & 2.30 & 27.49 & 75.86 \\
        BM25 + Ollama & 2.30 & 14.40 & 75.86 \\
        Iter-RetGen + Ollama & 2.30 & 12.96 & 72.41 \\
        Adaptive-RAG + Ollama & 0.00 & 12.71 & 8.05 \\
        \bottomrule
    \end{tabular}
\end{table}

\subsection{Key Findings from Baselines}

\textbf{Finding 1: Dense retrieval outperforms BM25.}

\begin{table}[H]
    \centering
    \caption{Retrieval method comparison}
    \label{tab:retrieval-comparison}
    \begin{tabular}{lccc}
        \toprule
        Metric & BM25 & Dense E5 & Improvement \\
        \midrule
        Recall@5 & 75.86\% & 83.91\% & +8.05\% \\
        Recall@10 & 80.46\% & 87.36\% & +6.90\% \\
        Best F1 & 27.49\% & 31.10\% & +3.61\% \\
        \bottomrule
    \end{tabular}
\end{table}

Dense retrieval's semantic understanding is crucial for academic terminology.

\textbf{Finding 2: Generator quality is the primary performance driver.}

With identical BM25 retrieval (75.86\% recall):
\begin{itemize}
    \item Ollama: 14.40\% F1
    \item GPT-4o-mini: 27.49\% F1 (+90.9\% improvement)
\end{itemize}

Upgrading the generator provides $\sim$3.6$\times$ more F1 gain than upgrading retrieval.

\textbf{Finding 3: Advanced RAG methods are limited by local LLM quality.}

Iter-RetGen and Adaptive-RAG performed worse than Naive RAG with Ollama, suggesting these methods require capable LLMs.

\textbf{Finding 4: TopK affects precision-recall trade-off.}

\begin{table}[H]
    \centering
    \caption{TopK ablation study}
    \label{tab:topk-ablation}
    \begin{tabular}{cccc}
        \toprule
        TopK & Recall (\%) & F1 (\%) & EM (\%) \\
        \midrule
        3 & 78.16 & 30.08 & \textbf{4.60} \\
        5 & 83.91 & 30.82 & 3.45 \\
        10 & \textbf{87.36} & \textbf{31.10} & 3.45 \\
        \bottomrule
    \end{tabular}
\end{table}

\section{RL Training Experiments}

\subsection{Initial Training: The Lazy Agent Problem}

We first trained the policy without the lazy agent fixes:

\begin{table}[H]
    \centering
    \caption{Initial RL training results (before fixes)}
    \label{tab:initial-rl}
    \begin{tabular}{ccccc}
        \toprule
        Epoch & Train F1 & Train Retr Rate & Val F1 & Val Retr Rate \\
        \midrule
        1 & 0.158 & 72.6\% & 0.141 & 0.0\% \\
        2 & 0.172 & 44.1\% & 0.141 & 0.0\% \\
        3 & 0.173 & 35.2\% & 0.141 & 0.0\% \\
        4 & 0.177 & 26.0\% & 0.141 & 0.0\% \\
        5 & 0.175 & 28.5\% & 0.141 & 0.0\% \\
        6 & 0.167 & 31.1\% & 0.141 & 0.0\% \\
        7 & 0.190 & 40.0\% & 0.141 & 0.0\% \\
        \bottomrule
    \end{tabular}
\end{table}

\textbf{Observation:} Training showed exploration (26--72\% retrieval due to epsilon-greedy), but the deterministic evaluation policy consistently chose ``never retrieve'' (0\% retrieval rate), resulting in validation F1 stuck at 0.141 (LLM parametric knowledge only).

\subsection{Training with Lazy Agent Fixes}

After implementing the five solutions (reward shaping, curriculum learning, entropy regularization, temperature sampling, difficulty features), we retrained:

\begin{table}[H]
    \centering
    \caption{RL training results after lazy agent fixes}
    \label{tab:fixed-rl}
    \begin{tabular}{cccccc}
        \toprule
        Epoch & Train F1 & Train Reward & Train Retr & Val F1 & Val Retr \\
        \midrule
        1 & 0.29 & 0.24 & 100\% & 0.34 & 100\% \\
        2 & 0.31 & 0.26 & 100\% & 0.34 & 100\% \\
        3 & 0.32 & 0.27 & 100\% & 0.34 & 100\% \\
        4 & 0.33 & 0.28 & 100\% & 0.34 & 100\% \\
        5 & 0.33 & 0.28 & 100\% & 0.34 & 100\% \\
        6 & 0.33 & 0.28 & 100\% & 0.34 & 100\% \\
        7 & 0.33 & 0.28 & 100\% & 0.34 & 100\% \\
        8 & 0.33 & 0.29 & 100\% & 0.34 & 100\% \\
        \bottomrule
    \end{tabular}
\end{table}

\textbf{Key Metrics (Final):}
\begin{itemize}
    \item Validation F1: 34.4\% (exceeds baseline 31.1\%)
    \item Validation Retrieval Rate: 100\%
    \item Lazy Agent Failures: 0
    \item Average Retrieval Probability: 99.6\%
\end{itemize}

\subsection{Before vs. After Comparison}

\begin{table}[H]
    \centering
    \caption{Impact of lazy agent fixes}
    \label{tab:before-after}
    \begin{tabular}{lccc}
        \toprule
        Metric & Before Fix & After Fix & Change \\
        \midrule
        Val F1 & 14.1\% & 34.4\% & +143\% \\
        Val Retrieval Rate & 0\% & 100\% & Fixed \\
        Lazy Agent Failures & 100\% & 0\% & Eliminated \\
        Comparison to Baseline & 45\% & 101\% & +56\% \\
        \bottomrule
    \end{tabular}
\end{table}

\section{Final Evaluation}

\subsection{Evaluation Setup}

We evaluated the trained policy against baselines on the full test set (87 samples):

\begin{lstlisting}[language=bash]
python evaluate_rl_agent.py --mode trained \
    --checkpoint best_model.pt --samples 87
\end{lstlisting}

\subsection{Results}

\begin{table}[H]
    \centering
    \caption{Final evaluation results (87 test samples)}
    \label{tab:final-eval}
    \begin{tabular}{lcccc}
        \toprule
        Method & F1 (\%) & EM (\%) & Reward & Retr Rate \\
        \midrule
        Trained RL Policy & 33.87 & 2.30 & 0.2847 & 100\% \\
        Baseline (Always Retrieve) & 34.08 & 3.45 & 0.2868 & 100\% \\
        No Retrieval (Pure LLM) & 18.52 & 0.00 & -0.0097 & 0\% \\
        \bottomrule
    \end{tabular}
\end{table}

\subsection{Analysis}

\textbf{Finding 1: Trained policy matches baseline.}

The RL-trained policy achieves 33.87\% F1, within 0.21\% of the always-retrieve baseline (34.08\% F1). This difference is within statistical noise.

\textbf{Finding 2: Policy learned correct domain behavior.}

For academic question-answering, the policy correctly learned that retrieval is almost always necessary:
\begin{itemize}
    \item Average retrieval probability: 99.6\%
    \item Actual retrieval rate: 100\%
\end{itemize}

\textbf{Finding 3: Retrieval is essential for this domain.}

The no-retrieval baseline achieves only 18.52\% F1, confirming that the LLM's parametric knowledge is insufficient for academic domain questions. The policy's decision to always retrieve is optimal.

\textbf{Finding 4: Lazy agent problem is solved.}

Unlike initial experiments where the policy learned to never retrieve, the fixed policy maintains 100\% retrieval with high F1, demonstrating that the five solutions successfully addressed the problem.

\section{Efficiency Analysis}

\subsection{Retrieval Rate vs. Performance Trade-off}

From earlier experiments with simulated retrieval policies:

\begin{table}[H]
    \centering
    \caption{Efficiency-quality trade-off (simulated policies)}
    \label{tab:efficiency}
    \begin{tabular}{ccccc}
        \toprule
        Retrieval Rate & F1 (\%) & EM (\%) & Efficiency Gain & F1 Retention \\
        \midrule
        100\% (baseline) & 34.56 & 4.60 & 0\% & 100\% \\
        80\% & 29.84 & 4.60 & 20\% & 86\% \\
        60\% & 26.08 & 3.45 & 40\% & 75\% \\
        40\% & 23.95 & 2.30 & 60\% & 69\% \\
        \bottomrule
    \end{tabular}
\end{table}

\textbf{Insight:} For datasets where retrieval is not always necessary, an 80\% retrieval policy could maintain 86\% of baseline F1 with 20\% fewer retrievals.

\subsection{API Cost Analysis}

\begin{table}[H]
    \centering
    \caption{API cost breakdown}
    \label{tab:api-cost}
    \begin{tabular}{ll}
        \toprule
        Component & Cost \\
        \midrule
        GPT-4o-mini input & \$0.15 / 1M tokens \\
        GPT-4o-mini output & \$0.60 / 1M tokens \\
        Training run (10 epochs, 492 samples) & $\sim$\$2--5 \\
        Evaluation (87 samples) & $\sim$\$0.50 \\
        \bottomrule
    \end{tabular}
\end{table}

\section{HotpotQA Comparison}

To contextualize our results, we compared performance on HotpotQA (a standard multi-hop QA benchmark):

\begin{table}[H]
    \centering
    \caption{Custom dataset vs. HotpotQA comparison}
    \label{tab:hotpotqa-comparison}
    \begin{tabular}{lcc}
        \toprule
        Metric & Custom Dataset & HotpotQA \\
        \midrule
        Best EM & 4.60\% & 19.30\% \\
        Best F1 & 31.10\% & 30.62\% \\
        Retrieval Recall@5 & 75.86\% (BM25) & 45.50\% \\
        \bottomrule
    \end{tabular}
\end{table}

\textbf{Insight:} HotpotQA achieves higher EM due to standardized Wikipedia answers, while F1 is similar. The custom dataset has higher retrieval recall because single-hop academic questions are easier to retrieve for than multi-hop reasoning questions.

\section{Summary}

Experiments demonstrated:

\begin{enumerate}
    \item \textbf{Baselines:} Dense E5 + GPT-4o-mini achieves 31.10\% F1, with generator quality being the primary performance driver.

    \item \textbf{Lazy Agent:} Initial RL training produced degenerate policies that never retrieve. Five complementary fixes resolved this.

    \item \textbf{Final Performance:} The trained policy achieves 33.87\% F1, matching the baseline and correctly learning that academic QA requires retrieval.

    \item \textbf{Efficiency Potential:} On datasets with mixed question difficulty, RL policies could reduce retrievals while maintaining most quality.
\end{enumerate}

% ============================================================================
% Chapter 6: Discussion
% ============================================================================

\chapter{Discussion}
\label{ch:discussion}

This chapter analyzes the experimental findings, discusses limitations, and considers implications for RAG system optimization.

\section{Interpretation of Results}

\subsection{Why Did the Policy Learn ``Always Retrieve''?}

The trained policy converged to a near-deterministic strategy of always retrieving (99.6\% retrieval probability). This behavior is optimal for our dataset for several reasons:

\textbf{1. Domain Characteristics:}
Academic question-answering requires specialized knowledge not present in the LLM's training data. The no-retrieval baseline (18.52\% F1) demonstrates that GPT-4o-mini's parametric knowledge is insufficient for this domain.

\textbf{2. Reward Signal:}
The wrong-no-retrieval penalty ($-0.3$) is applied when F1 $< 0.3$, which occurs almost always without retrieval on this dataset. The expected value calculation favors retrieval:
\begin{align}
    \mathbb{E}[\text{Retrieve}] &= 0.34 - 0.1 = 0.24 \\
    \mathbb{E}[\text{No Retrieve}] &= 0.18 - 0.3 = -0.12
\end{align}

\textbf{3. Correct Learning:}
The policy discovered through training what would otherwise need to be hardcoded as a rule: ``always retrieve for academic questions.'' This demonstrates that RL can learn domain-appropriate behavior from data.

\subsection{The Value of the Lazy Agent Fix}

The lazy agent problem and its resolution provide several insights:

\textbf{Reward Shaping Matters:}
The original reward function inadvertently created an incentive to skip retrieval. The fix required making the penalty for wrong no-retrieval answers ($p_{\text{lazy}} = 0.3$) exceed the retrieval cost ($c_{\text{retr}} = 0.1$). This principle---that negative outcomes from inaction should be penalized more than the cost of action---has broader applicability in RL reward design.

\textbf{Curriculum Learning Helps:}
Forcing high retrieval rates early in training ensured the policy experienced the benefits of retrieval before being allowed to skip it. Without curriculum learning, the policy could converge to local minima without exploring the retrieval action space.

\textbf{Deterministic Evaluation is Dangerous:}
The policy appeared to learn during training (with epsilon-greedy exploration) but collapsed to degenerate behavior during deterministic evaluation. Temperature-based sampling during evaluation prevented this collapse.

\subsection{Comparison to Related Work}

Our approach differs from prior work in several ways:

\begin{table}[H]
    \centering
    \caption{Comparison with related approaches}
    \label{tab:related-comparison}
    \begin{tabularx}{\textwidth}{lXX}
        \toprule
        Approach & Method & Key Difference \\
        \midrule
        Self-RAG & LLM outputs retrieval tokens & Requires fine-tuning the LLM itself \\
        Adaptive-RAG & Query complexity classifier & Uses heuristic rules, not learned policy \\
        FLARE & Confidence-based triggers & Requires token probabilities (not available for all APIs) \\
        Ours & Separate policy network with RL & Modular, works with black-box LLMs \\
        \bottomrule
    \end{tabularx}
\end{table}

Our modular approach allows the policy to be trained independently of the generator, enabling use with proprietary APIs like GPT-4o-mini where fine-tuning is not possible.

\section{Limitations}

\subsection{Dataset Homogeneity}

Our custom dataset consists entirely of academic questions that require retrieval. This homogeneity means:
\begin{itemize}
    \item The optimal policy is trivial (always retrieve)
    \item We cannot demonstrate selective retrieval behavior
    \item Results may not generalize to mixed-difficulty datasets
\end{itemize}

A dataset with questions of varying difficulty (some answerable from parametric knowledge, some requiring retrieval) would better showcase the policy's decision-making capability.

\subsection{Single-Step Decision}

Our MDP formulation uses a single retrieval decision per question. This limits:
\begin{itemize}
    \item Multi-hop reasoning (no iterative retrieval)
    \item Query refinement based on initial results
    \item Adaptive topk selection
\end{itemize}

Extending to multi-step decisions would increase complexity but enable more sophisticated retrieval strategies.

\subsection{Computational Cost}

RL training requires:
\begin{itemize}
    \item Multiple forward passes through the retriever and generator
    \item API calls for each training sample per epoch
    \item Significant time (18+ hours for 10 epochs)
\end{itemize}

This limits rapid iteration and hyperparameter tuning.

\subsection{Evaluation Metrics}

We use F1 score as the primary metric, which:
\begin{itemize}
    \item Rewards partial matches
    \item May not capture semantic correctness
    \item Differs from human judgment of answer quality
\end{itemize}

Human evaluation or semantic similarity metrics could provide additional perspectives.

\section{Implications}

\subsection{For RAG System Design}

\textbf{1. Generator Quality is Primary:}
Our baseline experiments showed that upgrading the generator (Ollama $\rightarrow$ GPT-4o-mini) improved F1 by 90.9\%, while upgrading retrieval (BM25 $\rightarrow$ Dense E5) improved F1 by only 13.1\%. This suggests prioritizing generator quality in resource-constrained settings.

\textbf{2. Domain-Specific Retrieval Strategies:}
The optimal retrieval strategy depends on the domain. For academic QA, always retrieving is optimal. For general-purpose assistants, adaptive retrieval could reduce costs.

\textbf{3. Reward Function Design:}
Reward functions for retrieval policies must carefully balance:
\begin{itemize}
    \item Quality incentives (reward correct answers)
    \item Efficiency incentives (penalize unnecessary retrieval)
    \item Safety margins (penalize wrong answers without retrieval more than the cost of retrieving)
\end{itemize}

\subsection{For RL in NLP}

\textbf{1. Lazy Behavior is Common:}
The lazy agent problem likely affects many RL applications where inaction has lower immediate cost than action. Our five-solution framework may be applicable to other domains.

\textbf{2. Curriculum Learning for Exploration:}
Forcing exploration early in training helps policies learn the value of actions before learning to avoid them.

\textbf{3. Evaluation Protocol Matters:}
The gap between training (stochastic) and evaluation (deterministic) behavior can mask fundamental problems. Temperature-based evaluation or other stochastic protocols may be necessary.

\section{Future Directions}

\subsection{Mixed-Difficulty Datasets}

Creating or using datasets with questions of varying difficulty would enable:
\begin{itemize}
    \item Demonstration of selective retrieval
    \item Analysis of what question features predict retrieval need
    \item More realistic efficiency gains
\end{itemize}

\subsection{Multi-Step Retrieval}

Extending the action space to include:
\begin{itemize}
    \item Multiple retrieval iterations
    \item Query reformulation actions
    \item Adaptive topk selection
\end{itemize}

would enable more sophisticated retrieval strategies for complex questions.

\subsection{Transfer Learning}

Investigating whether policies trained on one domain transfer to others could reduce training costs and enable rapid deployment to new domains.

\subsection{Integration with Query Rewriting}

Our pipeline includes a query rewriter, but it is trained separately from the retrieval policy. Joint optimization could improve end-to-end performance.

\section{Summary}

This chapter analyzed experimental findings, identifying that the policy correctly learned domain-appropriate behavior (always retrieve for academic QA). The lazy agent problem highlighted important principles for reward design in retrieval optimization. Limitations include dataset homogeneity and single-step decisions, while implications span RAG system design and RL methodology. Future work should explore mixed-difficulty datasets and multi-step retrieval strategies.

% ============================================================================
% Chapter 7: Conclusion
% ============================================================================

\chapter{Conclusion}
\label{ch:conclusion}

\section{Summary of Contributions}

This dissertation investigated the use of Reinforcement Learning to optimize Retrieval-Augmented Generation systems. The key contributions are:

\textbf{1. RL-RAG Framework:}
We developed a modular pipeline integrating neural policy networks with RAG components. The framework uses REINFORCE for policy gradient optimization and supports curriculum learning, entropy regularization, and temperature-based sampling.

\textbf{2. Lazy Agent Problem and Solutions:}
We identified the ``lazy agent'' problem---where policies learn to never retrieve---and developed five complementary solutions:
\begin{itemize}
    \item Reward shaping with wrong-no-retrieval penalty
    \item Curriculum learning with phased minimum retrieval rates
    \item Entropy regularization for exploration
    \item Temperature-based soft sampling during evaluation
    \item Question difficulty features for informed decisions
\end{itemize}

\textbf{3. Custom Academic Dataset:}
We created a high-quality QA dataset of 492 training and 87 test samples from 2,000 arXiv papers, with multi-stage quality filtering achieving $<5\%$ false positive rate.

\textbf{4. Comprehensive Baseline Study:}
We evaluated 8+ RAG configurations, finding that:
\begin{itemize}
    \item Dense retrieval (E5) outperforms sparse retrieval (BM25) by 8--11\% recall
    \item Generator quality (GPT-4o-mini vs. Ollama) is the primary performance driver (+90.9\% F1)
    \item The best baseline achieves 31.10\% F1 with Dense E5 + GPT-4o-mini
\end{itemize}

\textbf{5. Empirical Validation:}
The trained RL policy achieves 33.87\% F1, matching the always-retrieve baseline (34.08\% F1) and exceeding the no-retrieval baseline (18.52\% F1) by 83\%. The policy correctly learned that academic domain questions require retrieval.

\section{Research Questions Revisited}

\textbf{RQ1: Can an RL-trained policy learn to make effective retrieval decisions?}

Yes. The trained policy achieves performance comparable to the optimal fixed strategy (always retrieve). For our academic QA domain, the policy correctly learned that retrieval is almost always necessary, demonstrating that RL can discover domain-appropriate behavior through training.

\textbf{RQ2: What reward function design enables stable training?}

Stable training required:
\begin{enumerate}
    \item Wrong-no-retrieval penalty exceeding retrieval cost ($p_{\text{lazy}} > c_{\text{retr}}$)
    \item Curriculum learning to expose the policy to retrieval benefits early
    \item Entropy regularization to maintain exploration
    \item Soft evaluation to prevent deterministic collapse
\end{enumerate}

Without these elements, the policy converged to degenerate ``never retrieve'' behavior.

\textbf{RQ3: How does an RL-optimized policy compare to fixed baselines?}

On our academic QA dataset, the RL policy matches the always-retrieve baseline (33.87\% vs. 34.08\% F1). This is the optimal result for this domain---the policy learned that retrieval is essential. On datasets with mixed question difficulty, RL policies could potentially reduce retrieval costs while maintaining quality.

\section{Broader Impact}

This work contributes to the growing field of adaptive RAG systems. As LLMs become more widely deployed, efficient retrieval strategies become increasingly important for:

\begin{itemize}
    \item \textbf{Cost reduction:} Avoiding unnecessary API calls and vector searches
    \item \textbf{Latency improvement:} Skipping retrieval for simple queries
    \item \textbf{Quality improvement:} Learning when retrieval helps vs. hurts
\end{itemize}

The lazy agent problem and its solutions have implications beyond RAG, applying to any RL setting where inaction has lower immediate cost than action.

\section{Future Work}

Several directions warrant further investigation:

\textbf{1. Mixed-Difficulty Datasets:}
Evaluating on datasets where some questions can be answered without retrieval would demonstrate selective retrieval behavior.

\textbf{2. Multi-Step Retrieval:}
Extending the action space to support iterative retrieval, query refinement, and adaptive topk selection would enable more sophisticated strategies.

\textbf{3. Cross-Domain Transfer:}
Investigating whether policies transfer across domains could reduce training costs.

\textbf{4. Joint Optimization:}
Training the retrieval policy jointly with query rewriting could improve end-to-end performance.

\textbf{5. Human Evaluation:}
Complementing automatic metrics with human judgment would provide additional validation.

\section{Concluding Remarks}

This dissertation demonstrated that Reinforcement Learning can optimize RAG retrieval decisions, but careful reward design is essential to avoid degenerate behaviors. The lazy agent problem---where policies learn to avoid action---is a fundamental challenge that we addressed through reward shaping, curriculum learning, and exploration mechanisms.

For domain-specific applications like academic question-answering, the optimal strategy may be straightforward (always retrieve), but RL provides a principled way to discover this through data rather than manual rule design. For more general applications with mixed query difficulty, RL-optimized policies offer the potential for significant efficiency gains while maintaining answer quality.

The modular framework, dataset, and solutions developed in this work provide a foundation for future research on adaptive retrieval in RAG systems.


% ----------------------------------------------------------------------------
% BACK MATTER
% ----------------------------------------------------------------------------

% Bibliography
\printbibliography[heading=bibintoc,title={References}]

% Appendices
\appendix
% ============================================================================
% Appendix A: Code Listings
% ============================================================================

\chapter{Code Listings}
\label{app:code}

This appendix provides key code excerpts from the implementation.

\section{Reward Function}

\begin{lstlisting}[language=Python, caption=RAG Reward Calculator (reward.py)]
class RAGRewardCalculator:
    """
    Reward calculator for RL-RAG agent training.

    Implements the reward function:
    - Base reward from answer quality (F1 or EM)
    - Retrieval cost penalty
    - Efficiency bonus for correct answers without retrieval
    - Wrong non-retrieval penalty to prevent lazy agent
    """

    def __init__(
        self,
        retrieval_cost: float = 0.1,
        correct_no_retrieval_bonus: float = 0.1,
        wrong_no_retrieval_penalty: float = 0.3,
        use_f1: bool = True,
        f1_threshold_for_correct: float = 0.5,
        f1_threshold_for_bad: float = 0.3,
        format_bonus: float = 0.05
    ):
        self.retrieval_cost = retrieval_cost
        self.correct_no_retrieval_bonus = correct_no_retrieval_bonus
        self.wrong_no_retrieval_penalty = wrong_no_retrieval_penalty
        self.use_f1 = use_f1
        self.f1_threshold_for_correct = f1_threshold_for_correct
        self.f1_threshold_for_bad = f1_threshold_for_bad
        self.format_bonus = format_bonus

    def compute_reward(
        self,
        prediction: str,
        ground_truths: List[str],
        did_retrieve: bool,
        num_retrievals: int = 1
    ) -> Tuple[float, dict]:
        # Calculate quality scores
        f1_score = compute_f1(prediction, ground_truths)
        em_score = compute_exact_match(prediction, ground_truths)
        quality_score = f1_score if self.use_f1 else em_score

        # Determine if answer is correct or bad
        is_correct = f1_score >= self.f1_threshold_for_correct
        is_bad = f1_score < self.f1_threshold_for_bad

        # Base quality reward
        reward = quality_score

        # Retrieval cost/bonus
        if did_retrieve:
            reward -= self.retrieval_cost * num_retrievals
        else:
            if is_correct:
                reward += self.correct_no_retrieval_bonus
            elif is_bad:
                # Penalize bad no-retrieval more than retrieval cost
                reward -= self.wrong_no_retrieval_penalty

        # Format bonus
        if prediction and 0 < len(prediction.strip()) < 500:
            reward += self.format_bonus

        return reward, {"f1": f1_score, "em": em_score}
\end{lstlisting}

\section{Policy Network}

\begin{lstlisting}[language=Python, caption=Retrieval Policy Network (enhanced\_pipeline.py)]
class RetrievalPolicyNetwork(nn.Module):
    """Neural network for retrieval decisions."""

    def __init__(self, input_dim: int = 768, hidden_dim: int = 256):
        super().__init__()
        self.network = nn.Sequential(
            nn.Linear(input_dim, hidden_dim),
            nn.ReLU(),
            nn.Linear(hidden_dim, 64),
            nn.ReLU(),
            nn.Linear(64, 1),
            nn.Sigmoid()
        )

    def forward(self, x: torch.Tensor) -> torch.Tensor:
        return self.network(x)

    def get_action(
        self,
        x: torch.Tensor,
        deterministic: bool = False,
        temperature: float = 1.0
    ) -> Tuple[bool, torch.Tensor, torch.Tensor]:
        """Get action with optional temperature scaling."""
        prob = self.forward(x)

        # Apply temperature for soft sampling
        if not deterministic and temperature != 1.0:
            logit = torch.log(prob / (1 - prob + 1e-8))
            prob = torch.sigmoid(logit / temperature)

        # Sample action
        if deterministic:
            action = prob > 0.5
        else:
            action = torch.bernoulli(prob)

        # Compute log probability
        log_prob = torch.log(prob + 1e-8) if action else \
                   torch.log(1 - prob + 1e-8)

        # Compute entropy for regularization
        entropy = -(prob * torch.log(prob + 1e-8) +
                   (1 - prob) * torch.log(1 - prob + 1e-8))

        return bool(action.item()), log_prob, entropy
\end{lstlisting}

\section{Training Loop}

\begin{lstlisting}[language=Python, caption=Policy Update with Entropy Bonus]
def update_policy(self, log_probs, rewards, entropies):
    """Update policy using REINFORCE with entropy bonus."""

    # Convert to tensors
    log_probs = torch.stack(log_probs)
    rewards = torch.tensor(rewards, dtype=torch.float32)
    entropies = torch.stack(entropies)

    # Compute advantages with baseline
    baseline = rewards.mean()
    advantages = rewards - baseline

    # Normalize advantages
    if len(advantages) > 1:
        advantages = (advantages - advantages.mean()) / \
                     (advantages.std() + 1e-8)

    # Policy gradient loss with entropy bonus
    policy_loss = -(log_probs * advantages).mean()
    entropy_loss = -self.entropy_coef * entropies.mean()
    total_loss = policy_loss + entropy_loss

    # Update
    self.optimizer.zero_grad()
    total_loss.backward()
    torch.nn.utils.clip_grad_norm_(
        self.pipeline.policy.parameters(), 1.0
    )
    self.optimizer.step()

    return total_loss.item(), entropies.mean().item()
\end{lstlisting}

\section{Curriculum Learning}

\begin{lstlisting}[language=Python, caption=Curriculum Learning Implementation]
def get_curriculum_params(self, epoch: int, total_epochs: int,
                          num_phases: int = 3):
    """Get curriculum learning parameters for current epoch."""

    phase_length = total_epochs / num_phases
    current_phase = int(epoch / phase_length)

    # Minimum retrieval rates per phase
    min_rates = [0.8, 0.4, 0.0]  # Force high retrieval early

    if current_phase < len(min_rates):
        min_retrieval_rate = min_rates[current_phase]
    else:
        min_retrieval_rate = 0.0

    return {
        'phase': current_phase + 1,
        'min_retrieval_rate': min_retrieval_rate
    }
\end{lstlisting}

\section{F1 Score Computation}

\begin{lstlisting}[language=Python, caption=Token-level F1 Score (reward.py)]
def normalize_answer(s: str) -> str:
    """Normalize answer for comparison."""
    def remove_articles(text):
        return re.sub(r"\b(a|an|the)\b", " ", text)
    def white_space_fix(text):
        return " ".join(text.split())
    def remove_punc(text):
        exclude = set(string.punctuation)
        return "".join(ch for ch in text if ch not in exclude)
    def lower(text):
        return text.lower()

    return white_space_fix(remove_articles(remove_punc(lower(s))))

def compute_f1(prediction: str, ground_truths: List[str]) -> float:
    """Compute token-level F1 score."""
    max_f1 = 0.0
    pred_tokens = normalize_answer(prediction).split()

    for gt in ground_truths:
        gt_tokens = normalize_answer(gt).split()
        common = Counter(pred_tokens) & Counter(gt_tokens)
        num_same = sum(common.values())

        if num_same == 0:
            continue

        precision = num_same / len(pred_tokens)
        recall = num_same / len(gt_tokens)
        f1 = 2 * precision * recall / (precision + recall)
        max_f1 = max(max_f1, f1)

    return max_f1
\end{lstlisting}

% ============================================================================
% Appendix B: Detailed Results
% ============================================================================

\chapter{Detailed Results}
\label{app:results}

This appendix provides detailed experimental results and additional analyses.

\section{Complete Baseline Results}

\begin{table}[H]
    \centering
    \caption{All baseline experiments on custom dataset}
    \label{tab:all-baselines}
    \begin{tabular}{llcccc}
        \toprule
        Retrieval & Generator & TopK & EM (\%) & F1 (\%) & Recall (\%) \\
        \midrule
        Dense E5 & GPT-4o-mini & 10 & 3.45 & 31.10 & 87.36 \\
        Dense E5 & GPT-4o-mini & 5 & 3.45 & 30.82 & 83.91 \\
        Dense E5 & GPT-4o-mini & 3 & 4.60 & 30.08 & 78.16 \\
        BM25 & GPT-4o-mini & 5 & 2.30 & 27.49 & 75.86 \\
        BM25 & Ollama & 5 & 2.30 & 14.40 & 75.86 \\
        BM25 & Ollama & 10 & 0.00 & 3.43 & 80.46 \\
        BM25 (Iter-RetGen) & Ollama & 5 & 2.30 & 12.96 & 72.41 \\
        BM25 (Adaptive-RAG) & Ollama & 5 & 0.00 & 12.71 & 8.05 \\
        \bottomrule
    \end{tabular}
\end{table}

\section{RL Training Detailed Logs}

\subsection{Before Lazy Agent Fix}

\begin{table}[H]
    \centering
    \caption{Epoch-by-epoch training (before fix)}
    \label{tab:training-before}
    \begin{tabular}{ccccccc}
        \toprule
        Epoch & Train F1 & Train Reward & Train Retr & Val F1 & Val Retr & $\epsilon$ \\
        \midrule
        1 & 0.158 & 0.188 & 72.6\% & 0.141 & 0.0\% & 0.50 \\
        2 & 0.172 & 0.211 & 44.1\% & 0.141 & 0.0\% & 0.35 \\
        3 & 0.173 & 0.214 & 35.2\% & 0.141 & 0.0\% & 0.24 \\
        4 & 0.177 & 0.221 & 26.0\% & 0.141 & 0.0\% & 0.17 \\
        5 & 0.175 & 0.219 & 28.5\% & 0.141 & 0.0\% & 0.12 \\
        6 & 0.167 & 0.209 & 31.1\% & 0.141 & 0.0\% & 0.08 \\
        7 & 0.190 & --- & 40.0\% & 0.141 & 0.0\% & 0.06 \\
        \bottomrule
    \end{tabular}
\end{table}

\subsection{After Lazy Agent Fix}

\begin{table}[H]
    \centering
    \caption{Epoch-by-epoch training (after fix)}
    \label{tab:training-after}
    \begin{tabular}{cccccccc}
        \toprule
        Epoch & Train F1 & Reward & Retr & Entropy & Val F1 & Val Retr & Phase \\
        \midrule
        1 & 0.29 & 0.24 & 100\% & 0.025 & 0.34 & 100\% & 1 \\
        2 & 0.31 & 0.26 & 100\% & 0.025 & 0.34 & 100\% & 1 \\
        3 & 0.32 & 0.27 & 100\% & 0.025 & 0.34 & 100\% & 1 \\
        4 & 0.33 & 0.28 & 100\% & 0.025 & 0.34 & 100\% & 2 \\
        5 & 0.33 & 0.28 & 100\% & 0.025 & 0.34 & 100\% & 2 \\
        6 & 0.33 & 0.28 & 100\% & 0.025 & 0.34 & 100\% & 2 \\
        7 & 0.33 & 0.28 & 100\% & 0.025 & 0.34 & 100\% & 3 \\
        8 & 0.33 & 0.29 & 100\% & 0.025 & 0.34 & 100\% & 3 \\
        \bottomrule
    \end{tabular}
\end{table}

\section{Hyperparameter Configuration}

\begin{table}[H]
    \centering
    \caption{Complete hyperparameter settings}
    \label{tab:hyperparams}
    \begin{tabular}{llc}
        \toprule
        Category & Parameter & Value \\
        \midrule
        \multirow{4}{*}{Data} & Dataset & custom \\
        & Training samples & 492 \\
        & Validation samples & 87 \\
        & Random seed & 42 \\
        \midrule
        \multirow{5}{*}{Training} & Epochs & 10 \\
        & Initial epsilon & 0.5 \\
        & Epsilon decay & 0.7 \\
        & Update frequency & 5 \\
        & Learning rate & 0.001 \\
        \midrule
        \multirow{5}{*}{Reward} & Retrieval cost & 0.1 \\
        & Efficiency bonus & 0.1 \\
        & Wrong no-retrieval penalty & 0.3 \\
        & Correct threshold & 0.5 \\
        & Bad threshold & 0.3 \\
        \midrule
        \multirow{3}{*}{Exploration} & Entropy coefficient & 0.01 \\
        & Eval temperature & 0.7 \\
        & Curriculum phases & 3 \\
        \midrule
        \multirow{3}{*}{Model} & Policy hidden dim & 256 \\
        & Encoder & E5-base-v2 \\
        & Generator & GPT-4o-mini \\
        \bottomrule
    \end{tabular}
\end{table}

\section{Dataset Quality Statistics}

\begin{table}[H]
    \centering
    \caption{QA pair quality assessment results}
    \label{tab:qa-quality}
    \begin{tabular}{lcc}
        \toprule
        Metric & Training Set & Test Set \\
        \midrule
        Original pairs & 1,000 & 200 \\
        After pre-filtering & 750 & 150 \\
        High quality ($\geq 8.5$) & 380 & 65 \\
        Promoted (8.0--8.5) & 112 & 22 \\
        Final filtered & 492 & 87 \\
        \midrule
        Average score & 8.35 & 8.30 \\
        Min score & 8.00 & 8.00 \\
        Max score & 9.50 & 9.50 \\
        Std deviation & 0.42 & 0.45 \\
        \bottomrule
    \end{tabular}
\end{table}

\section{Retrieval Analysis}

\begin{table}[H]
    \centering
    \caption{Retrieval recall by method and TopK}
    \label{tab:retrieval-detail}
    \begin{tabular}{lccccc}
        \toprule
        Method & @1 & @3 & @5 & @10 & @20 \\
        \midrule
        BM25 & 32.18\% & 56.32\% & 75.86\% & 80.46\% & 85.06\% \\
        Dense E5 & 45.98\% & 78.16\% & 83.91\% & 87.36\% & 90.80\% \\
        \midrule
        Improvement & +13.80\% & +21.84\% & +8.05\% & +6.90\% & +5.74\% \\
        \bottomrule
    \end{tabular}
\end{table}

\section{Error Analysis}

\subsection{Common Error Types}

\begin{table}[H]
    \centering
    \caption{Error categorization (sample of 50 incorrect predictions)}
    \label{tab:errors}
    \begin{tabular}{lcc}
        \toprule
        Error Type & Count & Percentage \\
        \midrule
        Retrieval failure (answer not in docs) & 18 & 36\% \\
        Partial answer (missing details) & 14 & 28\% \\
        Wrong synthesis (incorrect reasoning) & 10 & 20\% \\
        Format mismatch (correct but different format) & 5 & 10\% \\
        Hallucination & 3 & 6\% \\
        \bottomrule
    \end{tabular}
\end{table}

\subsection{Performance by Question Type}

\begin{table}[H]
    \centering
    \caption{F1 score by question type}
    \label{tab:question-types}
    \begin{tabular}{lcc}
        \toprule
        Question Type & Count & Avg F1 (\%) \\
        \midrule
        What/Which & 35 & 34.2 \\
        How & 22 & 31.8 \\
        Why & 12 & 28.5 \\
        Who/Where/When & 10 & 38.1 \\
        Yes/No & 8 & 25.3 \\
        \bottomrule
    \end{tabular}
\end{table}

\section{Computational Resources}

\begin{table}[H]
    \centering
    \caption{Resource usage summary}
    \label{tab:resources}
    \begin{tabular}{lc}
        \toprule
        Resource & Usage \\
        \midrule
        Training time (10 epochs) & $\sim$18 hours \\
        GPU memory (MPS) & $\sim$4 GB \\
        FAISS index size & 128 MB \\
        Corpus size & 41,717 chunks \\
        API calls (training) & $\sim$5,000 \\
        API calls (evaluation) & $\sim$300 \\
        Estimated API cost & $\sim$\$5 total \\
        \bottomrule
    \end{tabular}
\end{table}


\end{document}
